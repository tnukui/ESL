\documentclass{jsarticle}


\usepackage{ascmac}
\usepackage[top=30truemm,bottom=30truemm,left=25truemm,right=25truemm]{geometry}
\usepackage{amsfonts}
\usepackage{amsmath,amssymb}
\usepackage[dvipdfmx]{graphicx}
\usepackage{cases}

\begin{document}

\title{The Elements of Statistical Learning\\統計的学習の基礎}
\author{nukui}
\date{\today}
\maketitle



\section{序章}

%%%%%%%%%%%%%%%%%%%%%%%%%%%%%%%%%%%%%%%%%%%%%%%%%%%%%%%%%%%%%%%%%%%%%%%%%%%%%
%%%%%%%%%%%%%%%%%%%%%%%%%%%%%%%%%%%%%%%%%%%%%%%%%%%%%%%%%%%%%%%%%%%%%%%%%%%%%
%%%%%%%%%%                        Chapter 2                        %%%%%%%%%%
%%%%%%%%%%%%%%%%%%%%%%%%%%%%%%%%%%%%%%%%%%%%%%%%%%%%%%%%%%%%%%%%%%%%%%%%%%%%%
%%%%%%%%%%%%%%%%%%%%%%%%%%%%%%%%%%%%%%%%%%%%%%%%%%%%%%%%%%%%%%%%%%%%%%%%%%%%%
\section{教師あり学習の概要}
%%% Ex.2.1
\subsection{}
\begin{shadebox}
$K$クラス分類問題の目標変数$t_k$が、第$k$要素のみが$1$で他の要素は$0$である$K$次元ベクトルによって表されているとする。予測結果$\hat{y}$を全ての要素の和が$1$となるように正規化したとき、$\hat{y}$の最大要素を持つクラスへの分類と$||t_k-\hat{y}||$を最小化するクラスへの分類が等価であることを示せ。
\end{shadebox}
$\hat{y}$の最大要素が第$j$要素$\hat{y_j}$であるとする。
\begin{align*}
&\ \ \ ||t_k-\hat{y}||^2-||t_j-\hat{y}||^2\\
&=\{\sum_{i\neq k}\hat{y_i}^2+(1-\hat{y_k})^2\}-\{\sum_{i\neq j}\hat{y_i}^2+(1-\hat{y_j})^2\}\\
&=\hat{y_j}^2+(1-\hat{y_k})^2-\hat{y_k}^2-(1-\hat{y_j})^2\\
&=2(\hat{y_j}-\hat{y_k})\geq0
\end{align*}
よって、$k=j$のとき、かつその時に限り、$||t_k-\hat{y}||$が最小値$||t_j-\hat{y}||$になることがわかる。以上より、$\hat{y}$の最大要素を持つクラスへの分類$j$と、$||t_k-\hat{y}||$を最小化するクラスへの分類が等価であることがわかる。


%%% Ex 2.2
\subsection{}
\begin{shadebox}
図2.5の試行の例においてベイズ決定境界を求めよ。
\end{shadebox}
(青色クラス)2次元ガウス分布$N((1,0)^T,\bf{I})$から生成された10個の平均ベクトル(青色クラス)を$\{m_1,m_2,...,m_{10}\}$とし、2次元ガウス分布$N((0,1)^T,\bf{I})$から生成された10個の平均ベクトル(オレンジ色クラス)を$\{n_1,n_2,...,n_{10}\}$とする。
\subsubsection{}
$\{m_1,m_2,...,m_{10}\}$と$\{n_1,n_2,...,n_{10}\}$の値がすでにわかっていると仮定する。
このとき、ベイズ決定境界上の点$x$の条件は
\[\Pr(青色|x)=\Pr(オレンジ色|x)\]
となる。
\[\frac{\Pr(青色|x)}{\Pr(オレンジ色|x)}=\frac{\Pr(青色|x)\Pr(x)}{\Pr(オレンジ色|x)\Pr(x)}=\frac{\Pr(x|青色)\Pr(青色)}{\Pr(x|オレンジ色)\Pr(オレンジ色)}\]
$\Pr(青色)=\Pr(オレンジ色)$なので、結局、ベイズ決定境界の式は
\[\Pr(x|青色)=\Pr(x|オレンジ色)\]
と表せる。
10個の平均ベクトルのどれが選ばれるかは等確率であり、$i$番目のベクトルが選ばれた時には、平均$m_i$(または$n_i$)で、分散$\bf{I}/5$の2変数ガウス分布に従うので、ベイズ決定境界上の点$x$の条件式は
\[\sum_{i=1}^{10}\frac{1}{10}\frac{\sqrt{5}}{2\pi}\exp\{-\frac{5(x-m_i)^T(x-m_i)}{2}\}=\sum_{i=1}^{10}\frac{1}{10}\frac{\sqrt{5}}{2\pi}\exp\{-\frac{5(x-n_i)^T(x-n_i)}{2}\}\]
と表される。
\subsubsection{}
$\{m_1,m_2,...,m_{10}\}$と$\{n_1,n_2,...,n_{10}\}$の値が未知だと仮定した場合、観測された値を頼りに確率を計算することができる。
$N$個の$p$次元ベクトル$x_i\ (i=1,...,N)$が青色の点として観測されていて、$y_i\ (i=1,...,N)$がオレンジ色の点として観測されているとする。平均$\mu$で、分散$\sigma$の2変数ガウス分布の$x$における確率密度関数を$f(x;\mu,\sigma)$と表すとすると、
\begin{align*}
&\ \ \Pr(x\ |\ \{x_1,x_2,...,x_N\},青色)\\
&=\sum_{m_1,...,m_{10}}\Pr(\{m_1,...,m_{10}\}\ | \{x_1,x_2,...,x_N\})\Pr(x|\{m_1,...,m_{10}\})\\
&=\sum_{m_1,...,m_{10}}\frac{\Pr(\{m_1,...,m_{10}\})\Pr(\{x_1,...,x_N\}|\{m_1,...,m_{10}\})}{\Pr(\{x_1,x_2,...,x_N\})}\Pr(x|\{m_1,...,m_{10}\})\\
&=\sum_{m_1,...,m_{10}}\frac{\{\prod_{i=1}^{10}f({m_i};(1,0)^T,{\bf I})\}\{\prod_{k=1}^N\sum_{j=1}^{10}\frac{1}{10}f(x_k;{m_j},{\bf I}/5)\}}
{\int\{\prod_{i=1}^{10}f({m_i}';(1,0)^T,{\bf I})\}\{\prod_{k=1}^N\sum_{j=1}^{10}\frac{1}{10}f(x_k;{m_j}',{\bf I}/5)\}d{m_1}'...d{m_{10}}'}
\Pr(x|\{m_1,...,m_{10}\})\\
&=\int\{\frac{\{\prod_{i=1}^{10}f({m_i};(1,0)^T,{\bf I})\}\{\prod_{k=1}^N\sum_{j=1}^{10}\frac{1}{10}f(x_k;{m_j},{\bf I}/5)\}\{\sum_{j=1}^{10}\frac{1}{10}f(x;m_j,{\bf I}/5)\}}
{\int\{\prod_{i=1}^{10}f({m_i}';(1,0)^T,{\bf I})\}\{\prod_{k=1}^N\sum_{j=1}^{10}\frac{1}{10}f(x_k;{m_j}',{\bf I}/5)\}d{m_1}'...d{m_{10}}'}\}dm_1...dm_{10}
\end{align*}
となる。また、ベイズ決定境界上の点$x$の条件式は
\[\Pr(x | \{x_1,x_2,...,x_N\},青色)=\Pr(x |\{y_1,y_2,...,y_N\},オレンジ色)\]
と表される。これを計算機でいい感じに求めることができるのかどうか、知りませんが。。。


%%% Ex.2.3
\subsection{}

\begin{shadebox}
式(2.24)を導け。
\[d(p,N)=(1-(\frac{1}{2})^{\frac{1}{N}})^{\frac{1}{p}}\]
\end{shadebox}
p次元空間の半径$r$の球の体積を$Cr^p$とおく($C$は定数)。半径1の球に$N$個の点が均一に散らばっているとすると、原点に最も近い点$x$が半径$r$の中に入っている確率は
\begin{align*}
\Pr(X<r)&=1-\Pr(X\geq r)\\
&=1-(\frac{C-Cr^p}{C})^N\\
&=1-(1-r^p)^N
\end{align*}
よって、$X$の中央値$d$は$\Pr(X<d)=\frac{1}{2}$となる境界なので、
\begin{align*}
1-(1-d^p)^N&=\frac{1}{2}\\
\frac{1}{2}&=(1-d^p)^N\\
1-d^p&=(\frac{1}{2})^\frac{1}{N}\\
d&=(1-(\frac{1}{2})^\frac{1}{N})^\frac{1}{p}
\end{align*}
以上より、$d(p,N)=(1-(\frac{1}{2})^\frac{1}{N})^\frac{1}{p}$と求められた。


%%%% Ex.2.4
\subsection{}
Later




%%%% Ex.2.5
\subsection{}
\begin{shadebox}
\begin{enumerate}
\item[(a)]
式(2.27)を導出せよ。
\item[(b)]
式(2.28)を導出せよ。
\end{enumerate}
\end{shadebox}

\begin{enumerate}
\item[(a)]
\begin{align*}
{\rm EPE}(x_0)&=\mathbb{E}_{y_0|x_0}[\mathbb{E}_{\cal T}[(y_0-\hat{y}_0)^2]]\\
&=\mathbb{E}_{y_0|x_0}[\mathbb{E}_{\cal T}[y_0^2-2y_0 \hat{y}_0+\hat{y}_0^2]]
\end{align*}
ここで、$y_0$はトレーニングデータ${\cal T}$には依存せず、予測値$\hat{y}_0$は真の値$y_0$には依存しないので、
\begin{align*}
{\rm EPE}(x_0)&=\mathbb{E}_{y_0|x_0}[y_0^2]-2\mathbb{E}_{y_0|x_0}[y_0] \mathbb{E}_{\cal T}[\hat{y}_0]+\mathbb{E}_{\cal T}[\hat{y}_0^2]
\end{align*}
分散と平均の関係$\mathbb{E}(x^2)={\rm Var(x)}+{\mathbb{E}(x)}^2$を用いることで、
\begin{align*}
{\rm EPE}(x_0)&={\rm Var}(y_0|x_0)+{\mathbb{E}_{y_0|x_0}[y_0]}^2+{\rm Var}_{\cal T}(\hat{y}_0)+{\mathbb{E}_{\cal T}[\hat{y}_0]}^2-2\mathbb{E}_{y_0|x_0}[y_0] \mathbb{E}_{\cal T}[\hat{y}_0]\\
&={\rm Var}(y_0|x_0)+{\rm Var}_{\cal T}(\hat{y}_0)+({\mathbb{E}_{y_0|x_0}[y_0]}-{\mathbb{E}_{\cal T}[\hat{y}_0]})^2
\end{align*}
ここで${\rm Var}(y_0|x_0)$は真の値の分散$\sigma^2$である。\\
$|\mathbb{E}_{y_0|x_0}[y_0]-\mathbb{E}_{\cal T}[\hat{y}_0]|$はBiasと呼ばれる値で、真の値の期待値$\mathbb{E}_{y_0|x_0}[y_0]$とモデルの予測した値の期待値$\mathbb{E}_{\cal T}[\hat{y}_0]$の間の乖離を示す。最小二乗推定は不偏であることが知られており、Biasは0になる。\\
\\
以下では、${\rm Var}_{\cal T}(\hat{y}_0)$を計算する。\\
トレーニングデータ${\cal T}$は入力${\bf X}$と出力$y$の組み合わせとして表現できるので、${\cal T}=({\bf X},y)$と書ける。
\begin{align*}
{\rm Var}_{\cal T}(\hat{y}_0)&=\mathbb{E}_{({\bf X},y)}[\hat{y}_0^2]-{\mathbb{E}_{({\bf X},y)}[\hat{y}_0]}^2\\
&=\int \{\Pr({\bf X},y)\hat{y}_0^2 \}d{\bf X}dy-(\int\{ \Pr({\bf X},y)\hat{y}_0 \}d{\bf X}dy)^2\\
&=\int \{\Pr({\bf X})(\int \{\Pr(y|{\bf X})\hat{y}_0^2 \}dy\})\}d{\bf X}-(\int \{\Pr({\bf X})(\int \{\Pr(y|{\bf X})\hat{y}_0 \}dy\})\}d{\bf X})^2\\
&=\int \{\Pr({\bf X})\mathbb{E}_{\Pr(y|{\bf X})}[\hat{y}_0^2]\}d{\bf X}-(\int \{\Pr({\bf X})\mathbb{E}_{\Pr(y|{\bf X})}[\hat{y}_0]\}d{\bf X})^2\\
&=\int \{ \Pr({\bf X})({\rm Var}_{\Pr(y|{\bf X})}(\hat{y}_0) +{\mathbb{E}_{\Pr(y|{\bf X})}[\hat{y}_0]}^2)\}d{\bf X}
-(\int \{\Pr({\bf X})\mathbb{E}_{\Pr(y|{\bf X})}[\hat{y}_0]\}d{\bf X})^2\\
&=\int \{\Pr({\bf X}) {\rm Var}_{\Pr(y|{\bf X})}(\hat{y}_0)\}d{\bf X} +
 \int\{\Pr({\bf X}){\mathbb{E}_{\Pr(y|{\bf X})}[\hat{y}_0]}^2\}d{\bf X}
-(\int \{\Pr({\bf X})\mathbb{E}_{\Pr(y|{\bf X})}[\hat{y}_0]\}d{\bf X})^2\\
&=\mathbb{E}_{{\bf X}}[{\rm Var}_{\Pr(y|{\bf X})}(\hat{y}_0)]
+\mathbb{E}_{{\bf X}}[(\mathbb{E}_{\Pr(y|{\bf X})}[\hat{y}_0])^2]
-(\mathbb{E}_{{\bf X}}[\mathbb{E}_{\Pr(y|{\bf X})}[\hat{y}_0]])^2\\
&=\mathbb{E}_{{\bf X}}[{\rm Var}_{\Pr(y|{\bf X})}(\hat{y}_0)]
+{\rm Var}_{{\bf X}}(\mathbb{E}_{\Pr(y|{\bf X})}[\hat{y}_0])
\end{align*}
以下では、$\mathbb{E}_{{\bf X}}[{\rm Var}_{\Pr(y|{\bf X})}(\hat{y}_0)]$と${\rm Var}_{{\bf X}}(\mathbb{E}_{\Pr(y|{\bf X})}[\hat{y}_0])$を計算していく。まず、$\mathbb{E}_{{\bf X}}[{\rm Var}_{\Pr(y|{\bf X})}(\hat{y}_0)]$を展開すると
\begin{align*}
\mathbb{E}_{{\bf X}}[{\rm Var}_{\Pr(y|{\bf X})}(\hat{y}_0)]&=\mathbb{E}_{{\bf X}}[{\rm Var}_{\Pr(y|{\bf X})}(x_0^T\hat{\beta})]\\
&=\mathbb{E}_{{\bf X}}[{\rm Var}_{\Pr(y|{\bf X})}(x_0^T ({\bf X}^T{\bf X})^{-1}{\bf X}^Ty)]\\
&=\mathbb{E}_{{\bf X}}[\mathbb{E}_{\Pr(y|{\bf X})}[(x_0^T ({\bf X}^T{\bf X})^{-1}{\bf X}^Ty)^2]-(\mathbb{E}_{\Pr(y|{\bf X})}[(x_0^T ({\bf X}^T{\bf X})^{-1}{\bf X}^Ty)])^2]\\
&=\mathbb{E}_{{\bf X}}[\mathbb{E}_{\Pr(y|{\bf X})}[(x_0^T ({\bf X}^T{\bf X})^{-1}{\bf X}^Ty)(x_0^T ({\bf X}^T{\bf X})^{-1}{\bf X}^Ty)^T]\\
&\ \ \ \ \ \ \ -(\mathbb{E}_{\Pr(y|{\bf X})}[(x_0^T ({\bf X}^T{\bf X})^{-1}{\bf X}^Ty)])(\mathbb{E}_{\Pr(y|{\bf X})}[(x_0^T ({\bf X}^T{\bf X})^{-1}{\bf X}^Ty)^T])]\\
&=\mathbb{E}_{{\bf X}}[\mathbb{E}_{\Pr(y|{\bf X})}[(x_0^T ({\bf X}^T{\bf X})^{-1}{\bf X}^Ty)(y^T{\bf X}({\bf X}^T{\bf X})^{-1}x_0)]\\
&\ \ \ \ \ \ \ -(\mathbb{E}_{\Pr(y|{\bf X})}[(x_0^T ({\bf X}^T{\bf X})^{-1}{\bf X}^Ty)])(\mathbb{E}_{\Pr(y|{\bf X})}[y^T{\bf X}({\bf X}^T{\bf X})^{-1}x_0])]\\
&=\mathbb{E}_{{\bf X}}[(x_0^T ({\bf X}^T{\bf X})^{-1}{\bf X}^T\mathbb{E}_{\Pr(y|{\bf X})}[yy^T]{\bf X}({\bf X}^T{\bf X})^{-1}x_0)\\
&\ \ \ \ \ \ \ -(x_0^T ({\bf X}^T{\bf X})^{-1}{\bf X}^T\mathbb{E}_{\Pr(y|{\bf X})}[y]\mathbb{E}_{\Pr(y|{\bf X})}[y^T]{\bf X}({\bf X}^T{\bf X})^{-1}x_0)]\\
&=\mathbb{E}_{{\bf X}}[(x_0^T ({\bf X}^T{\bf X})^{-1}{\bf X}^T(\mathbb{E}_{\Pr(y|{\bf X})}[yy^T]-\mathbb{E}_{\Pr(y|{\bf X})}[y]\mathbb{E}_{\Pr(y|{\bf X})}[y^T]){\bf X}({\bf X}^T{\bf X})^{-1}x_0)]
\end{align*}
観測値$y_i$が互いに独立で、分散$\sigma^2$を持つとすると$\mathbb{E}_{\Pr(y|{\bf X})}[yy^T]-\mathbb{E}_{\Pr(y|{\bf X})}[y]\mathbb{E}_{\Pr(y|{\bf X})}[y^T]=\sigma^2 I$なので、
\begin{align*}
\mathbb{E}_{{\bf X}}[{\rm Var}_{\Pr(y|{\bf X})}(\hat{y}_0)]&=\mathbb{E}_{{\bf X}}[(x_0^T ({\bf X}^T{\bf X})^{-1}{\bf X}^T(\sigma^2 I){\bf X}({\bf X}^T{\bf X})^{-1}x_0)]\\
&=\mathbb{E}_{{\bf X}}[x_0^T ({\bf X}^T{\bf X})^{-1}x_0\sigma^2]
\end{align*}
ここで、${\cal T}=({\bf X},y)$であり、またカッコの中に$y$に依存する項目が出てこないので、
\[\mathbb{E}_{{\bf X}}[{\rm Var}_{\Pr(y|{\bf X})}(\hat{y}_0)]=\mathbb{E}_{{\bf X}}[x_0^T ({\bf X}^T{\bf X})^{-1}x_0\sigma^2]=\mathbb{E}_{\cal T}[x_0^T ({\bf X}^T{\bf X})^{-1}x_0\sigma^2]\]
次に、${\rm Var}_{{\bf X}}(\mathbb{E}_{\Pr(y|{\bf X})}[\hat{y}_0])$を計算していく。
\begin{align*}
{\rm Var}_{{\bf X}}(\mathbb{E}_{\Pr(y|{\bf X})}[\hat{y}_0])&={\rm Var}_{{\bf X}}(\mathbb{E}_{\Pr(y|{\bf X})}[x_0^T\hat{\beta}])\\
&={\rm Var}_{{\bf X}}(\mathbb{E}_{\Pr(y|{\bf X})}[x_0^T({\bf X}^T{\bf X})^{-1}{\bf X}^Ty])\\
&={\rm Var}_{{\bf X}}(x_0^T({\bf X}^T{\bf X})^{-1}{\bf X}^T{\bf X}\beta)\\
&={\rm Var}_{{\bf X}}(x_0^T\beta)
\end{align*}
これは${\bf X}$に依存しない定数なので、${\rm Var}_{{\bf X}}(\mathbb{E}_{\Pr(y|{\bf X})}[\hat{y}_0])=0$
\\
結局、${\rm EPE}(x_0)=\sigma^2+\mathbb{E}_{\cal T}[x_0^T ({\bf X}^T{\bf X})^{-1}x_0\sigma^2]$という(2.27)式が導かれた。




\item[(b)]
$N$が大きく${\cal T}$がランダムに選択されており、$\mathbb{E}_{\cal T}(X)=0$であるとする\footnote{${\bf X}$は観測された値を行とする行列を表しており、$X$は一つの観測値の確率変数を表す。$X$は$p$次元のベクトル。}。${\bf X}^T{\bf X}\to N{\rm Cov}(X)$となるので、
\begin{align*}
\mathbb{E}_{x_0}[{\rm EPE}(x_0)]&=\mathbb{E}_{x_0}[x_0^T \mathbb{E}_{\cal T}[({\bf X}^T{\bf X})^{-1}]x_0]\sigma^2+\sigma^2\\
&\sim\mathbb{E}_{x_0}[x_0^T (N{\rm Cov}(X))^{-1}x_0]\sigma^2+\sigma^2\\
&=\mathbb{E}_{x_0}[x_0^T ({\rm Cov}(X))^{-1}x_0]\sigma^2/N+\sigma^2
\end{align*}
トレース演算の巡回性$[{\rm trace}(AB)={\rm trace}(BA)]$および線形性により
\begin{align*}
\mathbb{E}_{x_0}[{\rm EPE}(x_0)]&\sim{\rm trace}[\mathbb{E}_{x_0}[x_0^T ({\rm Cov}(X))^{-1}x_0]]\sigma^2/N+\sigma^2\\
&=\mathbb{E}_{x_0}[{\rm trace}[x_0^T ({\rm Cov}(X))^{-1}x_0]]\sigma^2/N+\sigma^2\\
&=\mathbb{E}_{x_0}[{\rm trace}[ ({\rm Cov}(X))^{-1}x_0x_0^T]]\sigma^2/N+\sigma^2\\
&={\rm trace}[ ({\rm Cov}(X))^{-1}\mathbb{E}_{x_0}[x_0x_0^T]]\sigma^2/N+\sigma^2
\end{align*}
$\mathbb{E}_{\cal T}[X]=0$より$\mathbb{E}_{x_0}[x_0]=0$である。$\mathbb{E}_{x_0}[x_0x_0^T]=\mathbb{E}_{x_0}[(x_0-\mathbb{E}_{x_0}[x_0])(x_0-\mathbb{E}_{x_0}[x_0])^T]={\rm Cov}(x_0)$となるので、
\begin{align*}
\mathbb{E}_{x_0}[{\rm EPE}(x_0)]&\sim{\rm trace}[ ({\rm Cov}(X))^{-1}{\rm Cov}(x_0)]\sigma^2/N+\sigma^2\\
&={\rm trace}[ ({\rm Cov}(X))^{-1}{\rm Cov}(x_0)]\sigma^2/N+\sigma^2\\
&={\rm trace}[{\rm I}]\sigma^2/N+\sigma^2\\
&=p\sigma^2/N+\sigma^2
\end{align*}
以上により、式(2.28)が導かれた。
\end{enumerate}







%%%% Ex.2.6
\subsection{}
Later 

%%%% Ex.2.7
\subsection{}
Later

%%%% Ex.2.8
\subsection{}
Later

%%%% Ex.2.9
\subsection{}
\begin{align*}
\mathbb{E}_{\{X,y\}}[R_{\rm tr}(\hat{\beta})]&=\mathbb{E}_{\{X,y\}}[\min_{\beta}R_{\rm tr}(\beta)]\\
&\leq\min_{\beta}\mathbb{E}_{\{X,y\}}[R_{\rm tr}(\beta)]\\
&=\min_{\beta}\mathbb{E}_{\{\tilde{X},\tilde{y}\}}[R_{\rm te}(\beta)]\\
&\leq \mathbb{E}_{\{\tilde{X},\tilde{y}\}}[R_{\rm te}(\hat{\beta})]
\end{align*}
上記のそれぞれの等号や不等号の変形を証明すれば良い。\\
\begin{enumerate}
\item
$\mathbb{E}_{\{X,y\}}[R_{\rm tr}(\hat{\beta})]=\mathbb{E}_{\{X,y\}}[\min_{\beta}R_{\rm tr}(\beta)]$は最小2乗法の定義より成り立つ。
\item
それぞれの値を$\beta$で最小化してから期待値(平均値)を取った方が、平均値を取った後に$\beta$で最小化するよりも小さいか等しくなるので、$\mathbb{E}_{\{X,y\}}[\min_{\beta}R_{\rm tr}(\beta)]\leq\min_{\beta}\mathbb{E}_{\{X,y\}}[R_{\rm tr}(\beta)]$が成り立つ。
\item
\begin{align*}
\min_{\beta}\mathbb{E}_{\{X,y\}}[R_{\rm tr}(\beta)]&=
\min_{\beta}\mathbb{E}_{\{X,y\}}[\frac{1}{N}\sum_{i=1}^{N}(y_i-\beta^{\mathrm{T}}x_i)^2]\\
&=\min_{\beta}\mathbb{E}_{\{x,y\}}[(y-\beta^{\mathrm{T}}x)^2]\\
&=\min_{\beta}\mathbb{E}_{\{\tilde{x},\tilde{y}\}}[(\tilde{y}-\beta^{\mathrm{T}}\tilde{x})^2]\\
&=\min_{\beta}\mathbb{E}_{\{\tilde{X},\tilde{y}\}}[\frac{1}{M}\sum_{i=1}^{M}(\tilde{y_i}-\beta^{\mathrm{T}}\tilde{x_i})^2]\\
&=\min_{\beta}\mathbb{E}_{\{\tilde{X},\tilde{y}\}}[R_{\rm te}(\beta)]
\end{align*}
より$\min_{\beta}\mathbb{E}_{\{X,y\}}[R_{\rm tr}(\beta)]=\min_{\beta}\mathbb{E}_{\{\tilde{X},\tilde{y}\}}[R_{\rm te}(\beta)]$が成り立つ。
\item
$\min_{\beta}$という関数の定義から$\min_{\beta}\mathbb{E}_{\{\tilde{X},\tilde{y}\}}[R_{\rm te}(\beta)]\leq \mathbb{E}_{\{\tilde{X},\tilde{y}\}}[R_{\rm te}(\hat{\beta})]$が成り立つ。
\end{enumerate}







%%%%%%%%%%%%%%%%%%%%%%%%%%%%%%%%%%%%%%%%%%%%%%%%%%%%%%%%%%%%%%%%%%%%%%%%%%%%%
%%%%%%%%%%%%%%%%%%%%%%%%%%%%%%%%%%%%%%%%%%%%%%%%%%%%%%%%%%%%%%%%%%%%%%%%%%%%%
%%%%%%%%%%                        Chapter 3                        %%%%%%%%%%
%%%%%%%%%%%%%%%%%%%%%%%%%%%%%%%%%%%%%%%%%%%%%%%%%%%%%%%%%%%%%%%%%%%%%%%%%%%%%
%%%%%%%%%%%%%%%%%%%%%%%%%%%%%%%%%%%%%%%%%%%%%%%%%%%%%%%%%%%%%%%%%%%%%%%%%%%%%
\section{回帰のための線形手法}
Later

%%%%%%%%%%%%%%%%%%%%%%%%%%%%%%%%%%%%%%%%%%%%%%%%%%%%%%%%%%%%%%%%%%%%%%%%%%%%%
%%%%%%%%%%%%%%%%%%%%%%%%%%%%%%%%%%%%%%%%%%%%%%%%%%%%%%%%%%%%%%%%%%%%%%%%%%%%%
%%%%%%%%%%                        Chapter 4                        %%%%%%%%%%
%%%%%%%%%%%%%%%%%%%%%%%%%%%%%%%%%%%%%%%%%%%%%%%%%%%%%%%%%%%%%%%%%%%%%%%%%%%%%
%%%%%%%%%%%%%%%%%%%%%%%%%%%%%%%%%%%%%%%%%%%%%%%%%%%%%%%%%%%%%%%%%%%%%%%%%%%%%
\section{分類のための線形手法}
Later

%%%%%%%%%%%%%%%%%%%%%%%%%%%%%%%%%%%%%%%%%%%%%%%%%%%%%%%%%%%%%%%%%%%%%%%%%%%%%
%%%%%%%%%%%%%%%%%%%%%%%%%%%%%%%%%%%%%%%%%%%%%%%%%%%%%%%%%%%%%%%%%%%%%%%%%%%%%
%%%%%%%%%%                        Chapter 5                        %%%%%%%%%%
%%%%%%%%%%%%%%%%%%%%%%%%%%%%%%%%%%%%%%%%%%%%%%%%%%%%%%%%%%%%%%%%%%%%%%%%%%%%%
%%%%%%%%%%%%%%%%%%%%%%%%%%%%%%%%%%%%%%%%%%%%%%%%%%%%%%%%%%%%%%%%%%%%%%%%%%%%%
\section{基底展開と正則化}

%%% Ex. 5.1
\subsection{}

\begin{shadebox}
式(5.3)中の切断べき基底関数が、文中で述べられていたような二つの節点を持つ3次スプラインの基底を表すことを示せ。
\end{shadebox}

まず、$h_1(X)=1,\ h_2(X)=X,\ h_3(X)=X^2,\ h_4(X)=X^3,\ h_5(X)=(X-\xi_1)_+^3,\ h_6(X)=(X-\xi_2)_+^3$と定義すると、任意の$\beta_m\ (m=1,2,..,6)$に対して、$f(X)=\sum_{m=1}^{M}\beta_m h_m(X)$とその1階導関数、および2階導関数がそれぞれ連続になることを示す。そのためには、各$i$に対して関数$h_i$とその1階導関数,2階導関数がそれぞれ連続になることを示せば十分。$h_1,h_2,h_3,h_4$は明らかに無限回微分可能である。$h_5$が$X=\xi_1$以外の点で連続であることは明らかである。よって、$X=\xi_1$の点で、$h_5,\ h_5', h_5''$がそれぞれ連続であることを確かめる。
\[\lim_{X\to\xi_1+0}h_5(X)=\lim_{X\to\xi_1+0}(X-\xi_1)^3=0\]
\[\lim_{X\to\xi_1-0}h_5(X)=\lim_{X\to\xi_1-0}0=0\]
よって、$h_5(X)$は$X=\xi_1$で連続である。また、
\[\lim_{X\to\xi_1+0}h_5'(X)=\lim_{X\to\xi_1+0}3(X-\xi_1)^2=0\]
\[\lim_{X\to\xi_1-0}h_5'(X)=\lim_{X\to\xi_1-0}0=0\]
よって、$h_5'(X)$は$X=\xi_1$で連続である。さらに、
\[\lim_{X\to\xi_1+0}h_5''(X)=\lim_{X\to\xi_1+0}6(X-\xi_1)=0\]
\[\lim_{X\to\xi_1-0}h_5''(X)=\lim_{X\to\xi_1-0}0=0\]
よって、$h_5''(X)$も$X=\xi_1$の点で連続である。同様に$h_6,\ h_6', h_6''$の連続性も証明できる。以上より、$f(X)$とその1階導関数、および2階導関数がそれぞれ連続になり、3次スプラインの条件を満たす。\\
\\
\\
次に、このように定義された$f$が3次スプラインの基底であることを確かめる。$h_1,h_2,h_3,h_4,h_5,h_6$が1次独立であることは明らかである\footnote{$\{h_n\}_{n=1}^6$の中のどの元も残りの元の線型結合で表現できない。}。よって、3次スプラインの条件を満たす任意の関数$g(X)$が$f(X)=\sum_{m=1}^{M}\beta_m h_m(X)$の形で表現できることを示せば十分である。
3次関数$g:(\xi_0,\xi_3)\rightarrow \mathbb{R}$が3次関数$g_1,g_2,g_3$を用いて各領域ごとに以下のように定義されるとしても$g$は一般性を失わない。
\[g(X)=
\begin{cases}
g_1(X)\quad(\xi_0\leq X\leq \xi_1)\\
g_1(X)+g_2(X)\quad(\xi_1\leq X\leq \xi_2)\\
g_1(X)+g_2(X)+g_3(X)\quad(\xi_2\leq X\leq \xi_3)\\
\end{cases}
\]
このとき、$g(X)$の連続性により、$g_2(\xi_1)=0,\ g_3(\xi_2)=0$が成り立つ。よって、
\begin{align*}
g_1(X)&=a X^3+ b X^2 + c X + d\\
g_2(X)&=e (X-\xi_1)(X^2 + hX+i) \\
g_3(X)&=j (X-\xi_2)(X^2 + kX + l)
\end{align*}
と表現できるはずである。さらに、$g'(X)$の連続性により、$g_2'(\xi_1)=0,\ g_3'(\xi_2)=0$が成り立つはずなので、
$e=0$または$\xi_1^2+h\xi_1+i=0$が成り立ち、さらに$j=0$または$\xi_2^2+k\xi_2+l=0$が成り立つ。よって、
\begin{align*}
g_1(X)&=a X^3+ b X^2 + c X + d\\
g_2(X)&=e (X-\xi_1)^2(X + m) \\
g_3(X)&=j (X-\xi_2)^2(X + n)
\end{align*}
と表現できることがわかった。さらに、$g''(X)$の連続性により、$g_2''(\xi_1)=0,\ g_3''(\xi_2)=0$が成り立つはずなので、
$e=0$または$\xi_1+m=0$が成り立ち、さらに$j=0$または$\xi_2+n=0$が成り立つ。
結局、
\begin{align*}
g_1(X)&=a X^3+ b X^2 + c X + d\\
g_2(X)&=e (X-\xi_1)^3 \\
g_3(X)&=j (X-\xi_2)^3
\end{align*}
と表現できることになり、この$g(X)$は関数$f(X)$と同じ形になった。以上より、関数$f$の形で任意の3次スプライン関数が表現できることになり、関数$h_1,...,h_6$は3次スプラインの基底であることがわかった。




%%% Ex. 5.2
\subsection{}
Later


%%% Ex. 5.3
\subsection{}
Later




%%% Ex. 5.4
\subsection{}
\begin{shadebox}
$K$個の内部節点を持つ3次スプラインの切断ベキ級数表現を考える。$f(X)$を次のように与える。
\[f(X)=\sum_{j=0}^3 \beta_j X^j + \sum_{k=1}^{K}\theta_k(X-\xi_k)_+^3\]

3次スプライン(5.2.1項)の自然境界条件が係数に関する次の線形制約を意味することを証明せよ。
\[\beta_2=0,\ \ \sum_{k=1}^{K}\theta_k=0\]
\[\beta_3=0,\ \ \sum_{k=1}^{K}\xi_k\theta_k=0\]
さらに、基底(5.4)および(5.5)を導け。
\end{shadebox}

$f(X)$を微分すると、以下のように表せる。
\begin{align*}
f'(X)&=\beta_1+2\beta_2 X+3\beta_3 X^2+3\sum_{k=1}^{K}\theta_k(X-\xi_k)_+^2\\
f''(X)&=2\beta_2 +6\beta_3 X+6\sum_{k=1}^{K}\theta_k(X-\xi_k)_+\\
f'''(X)&=6\beta_3+6\sum_{k=1}^{K}\theta_k I(\xi_k\leq X)
\end{align*}
3次スプライン$f(X)$の境界接点を$s,t\ \ (s<\xi_1,\ t>\xi_K)$とすると、3次自然スプラインの条件により$X=s, X=t$の点で$f$は線形になるはずなので、$f''(s)=0,\ f'''(s)=0,\ f''(t)=0,\ f'''(t)=0$を代入すれば、
\[\beta_2=0,\ \ \sum_{k=1}^{K}\theta_k=0\]
\[\beta_3=0,\ \ \sum_{k=1}^{K}\xi_k\theta_k=0\]
が得られる。\\
さて、任意の実数$\phi_1,\phi_2,...,\phi_{K-2}$について、
\begin{align*}
\theta_k&=\frac{\phi_k}{\xi_K-\xi_k}\ \ \ (k=1,2,...,K-2)\\
\theta_{K-1}&=-\frac{\sum_{k=1}^{K-2}\phi_k}{\xi_K-\xi_{K-1}}\\
\theta_K&=-\sum_{k=1}^{K-2}\frac{\phi_k}{\xi_K-\xi_k}+\frac{\sum_{k=1}^{K-2}\phi_k}{\xi_K-\xi_{K-1}}
\end{align*}
とおけば、$\sum_{k=1}^{K}\theta_k=0,\ \sum_{k=1}^{K}\xi_k\theta_k=0$を満たす。$\phi_1,\phi_2,...,\phi_{K-2}$は任意の実数で良かったので、任意の3次自然スプラインの係数$\theta_1,\theta_2,...,\theta_K$が表現できることもわかる。$\phi_k$を$N_{k+2}(X)$の係数と考えれば、式(5.4)と式(5,5)が導ける。


%%% Ex 5.5
\subsection{}
Later
%%% Ex 5.6
\subsection{}
Later

%%% Ex 5.7
\subsection{平滑化スプラインの導出}
\begin{shadebox}
$N\geq 2$とし、さらに$a<x_1<...<x_n< b$として組$\{x_i, z_i\}_{i=1}^{N}$を内挿する3次自然スプラインを$g$とする。これは、全ての$x_i$において節点を持つ自然スプラインであり、系列$z_i$を厳密に内挿する係数を決めることができるような$N$次元関数空間に含まれる。$N$組を内挿する$[a,b]$上の任意のその他の微分可能関数を$\tilde{g}$とする。
\begin{enumerate}
\item[(a)]
$h(x)=\tilde{g}(x)-g(x)$とする。データ上での積分と$g$が3次自然スプラインであるという事実を用いて、次式を示せ。
\begin{align*}
\int_{a}^{b}g''(x)h''(x)dx&=-\sum_{j=1}^{N-1}g'''(x_j^{+})\{h(x_{j+1})-h(x_j)\}\\
&=0
\end{align*}

\item[(b)]
次式を示せ。また、等式は$h$が$[a,b]$内で一様に$0$である場合のみ成り立つことを示せ。
\[\int_a^b \tilde{g}''(t)^2dt\geq\int_a^b g''(t)^2 dt\]

\item[(c)]
次式で表される罰則付き最小2乗問題を考える。
\[ \min_f [\sum_{i=1}^N (y_i-f(x_i))^2 + \lambda\int_a^b f''(t)^2dt]\]
(b)を用いて、この最小解が各$x_i$で節点を持つ3次スプラインでなければならないことを述べよ。
\end{enumerate}
\end{shadebox}


\begin{enumerate}
\item[(a)]部分積分することで、
\[\int_{a}^{b}g''(x)h''(x)dx=[g''(x)h'(x)]_a^b-\int_{a}^{b}g'''(x)h'(x)dx\]
となる。境界接点で$g$は線形になるので、$g''(a)$と$g''(b)$は$0$である。よって、
\begin{align*}
\int_{a}^{b}g''(x)h''(x)dx&=-\int_{a}^{b}g'''(x)h'(x)dx\\
&=-\{\int_{a}^{x_1}g'''(x)h'(x)dx+\sum_{j=1}^{N-1}\int_{x_j}^{x_{j+1}}g'''(x)h'(x)dx+\int_{x_N}^{b}g'''(x)h'(x)dx\}\\
&=-\{\sum_{j=1}^{N-1}[g'''(x)h(x)]_{x_j}^{x_{j+1}}\}\\
&=-\sum_{j=1}^{N-1}g'''(x_j^{+})\{h(x_{j+1})-h(x_j)\}\\
\end{align*}
再び部分積分を適用し、$g$は3次式のため、$\int g^{(4)}(x)h(x)dx$の項が消えた。また、$g$は3次自然スプラインのため、境界接点に接する区間$[a,x_1]$と$[x_N,b]$では$g'''(x)$が$0$になるため、$\int_{a}^{x_1}g'''(x)h'(x)dx$と$\int_{x_N}^{b}g'''(x)h'(x)dx$の項が消える。ここで、$x_j^{+}$という記号は$x_j$に微小量を足し合わせた値という意味で使用した\footnote{各節点$x=x_j$で$f'''(x)$が不連続のため、どちらの領域に属しているかを明示する必要がある。}。
$\tilde{g}(x)$と$g(x)$は各$x_j\ (j=1,...,n)$で同一の値$z_j$をとるので、$h(x_j)=0$となる。結局、
\[\int_{a}^{b}g''(x)h''(x)dx=0\]
が示された。

\item[(b)]
(a)より $\int_a^b h''(t)g''(t)dt=0$となるので、
\begin{align*}
\int_a^b \tilde{g}''(t)^2dt&=\int_a^b (h''(t)+g''(t))^2dt\\
&=\int_a^b h''(t)^2+2h''(t)g''(t)+g''(t)^2dt\\
&=\int_a^b h''(t)^2+g''(t)^2dt\\
&\geq \int_a^b g''(t)^2dt
\end{align*}
等号成立に必要かつ十分な条件は、$\int_a^b h''(t)^2dt=0$となることである。これは、$h$が$[a,b]$内で一様に$0$である場合のみ成り立つことを示している。

\item[(c)]
$\{x_i, z_i\}_{i=1}^{N}$を内挿する任意の関数で、第一項$\sum_{i=1}^N (y_i-f(x_i))^2$の部分が$0$になる。\\
第二項$\lambda\int_a^b f''(t)^2dt$については、3次自然スプライン$g$の方が、$N$組を内挿する$[a,b]$上の任意の微分可能関数$\tilde{g}$を使用した場合と同じかそれよりも小さくなることを(b)で示した。結局、罰則付き最小二乗問題の解は3次自然スプラインでなければならない。
\end{enumerate}



\end{document}