\documentclass{jsarticle}


\usepackage{ascmac}
\usepackage[top=30truemm,bottom=30truemm,left=25truemm,right=25truemm]{geometry}
\usepackage{amsfonts}
\usepackage{amsmath,amssymb}
\usepackage[dvipdfmx]{graphicx}
\usepackage{cases}

\begin{document}

\title{The Elements of Statistical Learning\\統計的学習の基礎}
\author{nukui}
\date{\today}
\maketitle



\section{序章}
\section{教師あり学習の概要}
%%% 2.1
\subsection{}
\begin{shadebox}
$K$クラス分類問題の目標変数$t_k$が、第$k$要素のみが$1$で他の要素は$0$である$K$次元ベクトルによって表されているとする。予測結果$\hat{y}$を全ての要素の和が$1$となるように正規化したとき、$\hat{y}$の最大要素を持つクラスへの分類と$||t_k-\hat{y}||$を最小化するクラスへの分類が等価であることを示せ。
\end{shadebox}
$\hat{y}$の最大要素が第$j$要素であるとする。
\begin{align*}
&\ \ \ ||t_k-\hat{y}||^2-||t_j-\hat{y}||^2\\
&=\{\sum_{i\neq k}\hat{y_i}^2+(1-\hat{y_k})^2\}-\{\sum_{i\neq j}\hat{y_i}^2+(1-\hat{y_j})^2\}\\
&=\hat{y_j}^2+(1-\hat{y_k})^2-\hat{y_k}^2-(1-\hat{y_j})^2\\
&=2(\hat{y_j}-\hat{y_k})
\end{align*}
よって、$k=j$のとき、$||t_k-\hat{y}||$が最小になることがわかる。以上より、$\hat{y}$の最大要素を持つクラスへの分類$j$と、$||t_k-\hat{y}||$を最小化するクラスへの分類が等価であることがわかる。


%%% 2.2
\subsection{}
\begin{shadebox}
図2.5の試行の例においてベイズ決定境界を求めよ。
\end{shadebox}
(青色クラス)2次元ガウス分布$N((1,0)^T,\bf{I})$から生成された10個の平均ベクトル(青色クラス)を$\{m_1,m_2,...,m_{10}\}$とし、2次元ガウス分布$N((0,1)^T,\bf{I})$から生成された10個の平均ベクトル(オレンジ色クラス)を$\{n_1,n_2,...,n_{10}\}$とする。
\subsubsection{}
$\{m_1,m_2,...,m_{10}\}$と$\{n_1,n_2,...,n_{10}\}$の値がすでにわかっていると仮定する。
このとき、ベイズ決定境界上の点$x$の条件は
\[\Pr(青色|x)=\Pr(オレンジ色|x)\]
となる。
\[\frac{\Pr(青色|x)}{\Pr(オレンジ色|x)}=\frac{\Pr(青色|x)\Pr(x)}{\Pr(オレンジ色|x)\Pr(x)}=\frac{\Pr(x|青色)\Pr(青色)}{\Pr(x|オレンジ色)\Pr(オレンジ色)}\]
$\Pr(青色)=\Pr(オレンジ色)$なので、結局、ベイズ決定境界の式は
\[\Pr(x|青色)=\Pr(x|オレンジ色)\]
と表せる。
10個の平均ベクトルのどれが選ばれるかは等確率であり、$i$番目のベクトルが選ばれた時には、平均$m_i$(または$n_i$)で、分散$\bf{I}/5$の2変数ガウス分布に従うので、ベイズ決定境界上の点$x$の条件式は
\[\sum_{i=1}^{10}\frac{1}{10}\frac{\sqrt{5}}{2\pi}\exp\{-\frac{5(x-m_i)^T(x-m_i)}{2}\}=\sum_{i=1}^{10}\frac{1}{10}\frac{\sqrt{5}}{2\pi}\exp\{-\frac{5(x-n_i)^T(x-n_i)}{2}\}\]
と表される。
\subsubsection{}
$\{m_1,m_2,...,m_{10}\}$と$\{n_1,n_2,...,n_{10}\}$の値が未知だと仮定した場合、観測された値を頼りに確率を計算することができる。
$N$個の$p$次元ベクトル$x_i\ (i=1,...,N)$が青色の点として観測されていて、$y_i\ (i=1,...,N)$がオレンジ色の点として観測されているとする。平均$\mu$で、分散$\sigma$の2変数ガウス分布の$x$における確率密度関数を$f(x;\mu,\sigma)$と表すとすると、
\begin{align*}
&\ \ \Pr(x\ |\ \{x_1,x_2,...,x_N\},青色)\\
&=\sum_{m_1,...,m_{10}}\Pr(\{m_1,...,m_{10}\}\ | \{x_1,x_2,...,x_N\})\Pr(x|\{m_1,...,m_{10}\})\\
&=\sum_{m_1,...,m_{10}}\frac{\Pr(\{m_1,...,m_{10}\})\Pr(\{x_1,...,x_N\}|\{m_1,...,m_{10}\})}{\Pr(\{x_1,x_2,...,x_N\})}\Pr(x|\{m_1,...,m_{10}\})\\
&=\sum_{m_1,...,m_{10}}\frac{\{\prod_{i=1}^{10}f({m_i};(1,0)^T,{\bf I})\}\{\prod_{k=1}^N\sum_{j=1}^{10}\frac{1}{10}f(x_k;{m_j},{\bf I}/5)\}}
{\int\{\prod_{i=1}^{10}f({m_i}';(1,0)^T,{\bf I})\}\{\prod_{k=1}^N\sum_{j=1}^{10}\frac{1}{10}f(x_k;{m_j}',{\bf I}/5)\}d{m_1}'...d{m_{10}}'}
\Pr(x|\{m_1,...,m_{10}\})\\
&=\int\{\frac{\{\prod_{i=1}^{10}f({m_i};(1,0)^T,{\bf I})\}\{\prod_{k=1}^N\sum_{j=1}^{10}\frac{1}{10}f(x_k;{m_j},{\bf I}/5)\}\{\sum_{j=1}^{10}\frac{1}{10}f(x;m_j,{\bf I}/5)\}}
{\int\{\prod_{i=1}^{10}f({m_i}';(1,0)^T,{\bf I})\}\{\prod_{k=1}^N\sum_{j=1}^{10}\frac{1}{10}f(x_k;{m_j}',{\bf I}/5)\}d{m_1}'...d{m_{10}}'}\}dm_1...dm_{10}
\end{align*}
となる。また、ベイズ決定境界上の点$x$の条件式は
\[\Pr(x | \{x_1,x_2,...,x_N\},青色)=\Pr(x |\{y_1,y_2,...,y_N\},オレンジ色)\]
と表される。これを計算機でいい感じに求めることができるのかどうか、知りませんが。。。


\subsection{}

\begin{shadebox}
式(2.24)を導け。
\[d(p,N)=(1-(\frac{1}{2})^{\frac{1}{N}})^{\frac{1}{p}}\]
\end{shadebox}
p次元空間の半径$r$の球の体積を$Cr^p$とおく($C$は定数)。半径1の球に$N$個の点が均一に散らばっているとすると、原点に最も近い点$x$が半径$r$の中に入っている確率は
\begin{align*}
\Pr(X<r)&=1-\Pr(X\geq r)\\
&=1-(\frac{C-Cr^p}{C})^N\\
&=1-(1-r^p)^N
\end{align*}
よって、$X$の中央値$d$は$\Pr(X<d)=\frac{1}{2}$となる境界なので、
\begin{align*}
1-(1-d^p)^N&=\frac{1}{2}\\
\frac{1}{2}&=(1-d^p)^N\\
1-d^p&=(\frac{1}{2})^\frac{1}{N}\\
d&=(1-(\frac{1}{2})^\frac{1}{N})^\frac{1}{p}
\end{align*}
以上より、$d(p,N)=(1-(\frac{1}{2})^\frac{1}{N})^\frac{1}{p}$と求められた。
























































\end{document}